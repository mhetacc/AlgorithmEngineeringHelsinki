\chapter{What Searching Is\label{intro}}

The act of "searching" has become so deeply ingrained in the modern society that we tend to take it for granted, not only assuming it normal to have immediate and easily accessible information on the tip of our thumbs, but expecting it: a study from 2004 showed that users were not willing to wait more than ten seconds for a page to load (\cite{waitTime}), fast forward twenty years and nowdays even a couple seconds holdup would be unacceptable, thus query retrieval needs to be fast. Blazingly fast in fact, since we need to account for all the delays typical of a gargantuan structure as big as the modern web, and, as the reader probably knows, it is \textit{not} a good idea to rely on memory's perfomances increasing over time: the smart way to tackle this problem is via research and development of efficient algorithms, and exactly which type should be self-evident from the title of this document. The problem of the set intersection constitutes the backbone of every query resolver in a (web) search engine, since every word in a query is interpreted as a collection of documents' IDs which contains it. \\
In this survey-style paper we will first explain what searching (i.e., querying) entails, show how a document (e.g., a web page) can be transformed into word tokens which are then further processed into inverted indexes, and, finally, we will see a collection of algorithms that concern themselves with intersecting sets, meaning finding common elements between two or more comparable collections. 
