% \begin{abstract}{finnish}

% Tämä dokumentti on tarkoitettu Helsingin yliopiston tietojenkäsittelytieteen osaston opin\-näyt\-teiden ja harjoitustöiden ulkoasun ohjeeksi ja mallipohjaksi. Ohje soveltuu kanditutkielmiin, ohjelmistotuotantoprojekteihin, seminaareihin ja maisterintutkielmiin. Tämän ohjeen lisäksi on seurattava niitä ohjeita, jotka opastavat valitsemaan kuhunkin osioon tieteellisesti kiinnostavaa, syvällisesti pohdittua sisältöä.


% Työn aihe luokitellaan  
% ACM Computing Classification System (CCS) mukaisesti, 
% ks.\ \url{https://dl.acm.org/ccs}. 
% Käytä muutamaa termipolkua (1--3), jotka alkavat juuritermistä ja joissa polun tarkentuvat luokat erotetaan toisistaan oikealle osoittavalla nuolella.

% \end{abstract}

\begin{otherlanguage}{english}
\begin{abstract}

The problem of \textit{set intersection} is a fundamental one in computer science, as it lies at the core of all those algorithms that concern themselves with finding documents based on keywords in a query, such as those used by search engines. In this document we explore both some fundamental search algorithms (necessary for the intersection of two lists), intersection algorithms that find common elements between two collections and, finally, melding algorithms that extend the problem of intersection to a collection of $k$ sets. We will also see how the proposed melding algorithms perform in practice, both on synthetic and real data.

\end{abstract}
\end{otherlanguage}
